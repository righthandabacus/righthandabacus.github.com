%% LyX 1.1 created this file.  For more info, see http://www.lyx.org/.
%% Do not edit unless you really know what you are doing.
\documentclass[12pt,a4paper,english]{article}
\usepackage{pslatex}
\usepackage[T1]{fontenc}
\usepackage[latin1]{inputenc}
\usepackage{babel}
\setlength\parskip{\medskipamount}
\setlength\parindent{0pt}
\usepackage{setspace}
\doublespacing

\makeatletter

%%%%%%%%%%%%%%%%%%%%%%%%%%%%%% LyX specific LaTeX commands.
\providecommand{\LyX}{L\kern-.1667em\lower.25em\hbox{Y}\kern-.125emX\@}

%%%%%%%%%%%%%%%%%%%%%%%%%%%%%% Textclass specific LaTeX commands.
 \newenvironment{lyxlist}[1]
   {\begin{list}{}
     {\settowidth{\labelwidth}{#1}
      \setlength{\leftmargin}{\labelwidth}
      \addtolength{\leftmargin}{\labelsep}
      \renewcommand{\makelabel}[1]{##1\hfil}}}
   {\end{list}}
 \newenvironment{lyxcode}
   {\begin{list}{}{
     \setlength{\rightmargin}{\leftmargin}
     \raggedright
     \setlength{\itemsep}{0pt}
     \setlength{\parsep}{0pt}
     \normalfont\ttfamily}%
    \item[]}
   {\end{list}}

\makeatother
\begin{document}

\title{Bayonne + Linux + Dialogic HOWTO}


\author{Adrian Sai-wah TAM (swtam@school.net.hk)\\
Hong Kong School Net, The Chinese University of Hong Kong}


\date{Version 0.2: September 2, 2002}

\maketitle

\section{Document History}

\begin{lyxlist}{00.00.0000}
\item [v0.2~(2002-09-02)]With the help of SR 5.1 for Linux Installation
HOWTO from Mr. Gerry Gilmore, enhanced the RedHat installation details.
Some tricks on Debian's installation is also added.
\item [v0.1~(2002-08-29)]First version
\end{lyxlist}

\section{Overview}

Dialogic is a subsidiary of Intel Corporation. It produces a series
of telephony devices for computers. The Dialogic devices are proud
of its generic software interface -- all Dialogic device share the
same API and device drivers (a.k.a. System Release), as said by Intel.
However, the Dialogic's API is not a very efficient tool for developing
telephony applications, especially in terms of the Rapid Application
Development tilde.

Bayonne, a software project of GNU, is aimed to provide a generic
framework for telephony hardware. It provides a high level abstraction
so that all different hardware uses the same script to do the same
thing (of course, provided that the hardware supports such features
required). The script that Bayonne uses is simple and easy to understand.
With the use of Bayonne, we can develop a telephony application easily.

Bayonne do not support all the telephony devices on the market. But
lucky, Dialogic series is one of the brand it supports. The following
describes how you can do with Dialogic and Bayonne on the Linux platform,
from the hardware set up to application development.


\section{Hardware installation}

The Dialogic devices usually have a board identification switch on
the board. The switch is something like a dial, with hexadecimal numbers
0 to F printed on it. This switch is for identifying and distinguishing
the board from others when you have more than one board installed
in a system. Every Dialogic board should have a unique identification
number, but if you don't care about which board and which board, you
can just make every board the identification number 0, which then,
automatic assignment is applied base on the PCI bus and slot number.
If you are not using the automatic assignment, remember to record
the identification number and the corresponding board models. These
are needed during the driver installation.


\section{Install System Release on Linux system}

The Linux system can be any distribution. But I would prefer classifying
the distributions into RedHat-alike and Debian-alike. Their difference
is somewhat large. But in this document, you only need to know one
thing: Do you use RPM for installation of new packages. If not, you
just read the Debian-alike part.


\subsection{Installation for RedHat-alike distributions}

I have tried the RedHat version 7.3 as well as Mandrake 9.0 beta 1
and beta 3. Installation on RedHat is easy. Just as the Intel says,
install Linux Stream package first, then do the Dialogic system release
installation. However, I found that installation on Mandrake is a
little bit difficult. I don't know it is the problem of Mandrake-modified
kernel or the instability of beta distributions. However, there is
still ways to solve it. Anyway, if you found that your RedHat-alike
distribution has some problem, do the tasks according to the paragraphs
followed. If you are RedHat, you can probably skip section \ref{KernUp}
and go from \ref{LiS}.


\subsubsection{\label{KernUp}Kernel upgrade}

The Dialogic system release version 5.1 is tailored for kernel version
2.4 series. Therefore, it would be best to keep up with the most current
2.4 kernel. I have not tried the installation on kernel 2.2 and the
development kernel 2.5, please tell me the result if you tried this.

You should first, go the \texttt{ftp://ftp.kernel.org/pub/linux/kernel/v2.4/}
and get the kernel tar ball. The one I'd tried is version 2.4.19.
After you get the tar ball (say, linux-2.4.19.tar.bz2), you extract
it into somewhere like \texttt{/usr/src}, like follows:

\begin{lyxcode}
cd~/var/tmp

wget~ftp://ftp.kernel.org/pub/linux/kernel/v2.4/linux-2.4.19.tar.bz2

cd~/usr/src

tar~jxf~/var/tmp/linux-2.4.19.tar.bz2
\end{lyxcode}
After extracting the tar ball, you should make the new kernel and
use it. It is quite easy to do even if it is your first time to compile
kernel. You just \texttt{chdir} into the kernel source directory and
than \texttt{make menuconfig}, choose the options you needed, than
\texttt{make bzImage}; \texttt{make modules}; \texttt{make modules\_install};
and finally, \texttt{make install}. I don't want to describe in too
detail, but you can consult The Linux Kernel HOWTO (\texttt{http://www.tldp.org/HOWTO/Kernel-HOWTO.html})
at Linux Documentation Project (\texttt{http://www.tldp.org/}) for
details on kernel compilation. The most important thing here is: enable
the Linux telephony support during you \texttt{make menuconfig}, either
in compiled-in or module is okay. After the compilation and installation,
remember to reboot and use the new kernel before proceed.


\subsubsection{Afraid of make bzImage?}

If you don't want to make your kernel from the tar ball, or you are
novice enough that do not know how to do it, or you have any reason
don't want to do it, you can also use the RedHat package manager for
the kernel upgrade purpose.

Dialogic System Release requires a clean and pure kernel. That means,
it would be better if you are using RedHat because RedHat seldom modify
the kernel. That was my experience that Mandrake 8.2 beta 1 has some
problem in the kernel that would give tons of harmless but noisy messages
during the SR's modules load. The System Release 5.1 requires kernel
version of 2.4.9+, therefore, if you are using RedHat 7.2, your kernel
must be upgraded. If you are using Red Hat 7.3, that's great! You
basically needs no upgrade of kernel. However, the make sure you have
the following packages installed:

\begin{itemize}
\item kernel-2.4.18-3, or kernel-smp-2.4.18-3 if you are using a system
with parallel processors
\item kernel-source-2.4.18-3
\item kernel-headers-2.4.18-3
\item some version of gcc, gmake, devel libraries like glibc-devel, runtime
libraries like glibc and libstdc++, ...
\end{itemize}
These packages are required because you need to compile something
with the kernel source code. You just need to {}``\texttt{rpm -i}''
or {}``\texttt{rpm -U}'' them. The Red Hat users of older version
may also use these RPM packages and upgrade your existing ones. Go
to http://www.redhat.com/ or http://www.rpmfind.net/ for details on
where to get these RPM packages. Again, if you have upgraded the kernel,
don't forget to re-run the lilo or grub and reboot before proceed!
RPM do not do these for you!


\subsubsection{\label{LiS}Install Linux Streams}

Linux Streams (LiS) is important to Dialogic System Release. It uses
LiS to make the driver for your system. So the first thing you have
to do is to go to: \texttt{ftp://ftp.gcom.com/pub/linux/src/LiS/}
and get the latest Linux Stream package. The lastest one at the time
I write this document is version 2.14.6. Same as downloading and installing
the kernel, you just download the tar ball (say, \texttt{LiS-2.14.6.tar.gz})
and extract everything into somewhere like \texttt{/usr/src} (recommended).

After that, you go to \texttt{/usr/src/LiS-2.14.6} and run \texttt{make}.
Then you will have an interactive configuration session which is same
as running \texttt{./configure}. Following is the questions you found:

\begin{enumerate}
\item How do you want to configure STREAMS?\\
It is recommended to run in Linux kernel. Although it is the choice
for running in user level, but it is recommended and required by the
Dialogic System Release to make it a kernel mode program.
\item Are you using the native Linux C compiler or are you crossing-compiling
using a different compiler?\\
Well...are you cross-compiling? I haven't tried and don't know whether
cross compiling works. Hereinafter, I assumed you are using a native
compiler, just as what I did.
\item Enter directory location of your kernel source\\
Usually the configuration program is smart enough to find the source
directory of the kernel that you are using. If it fails, tell it the
right one and press the enter key.
\item Which version of the kernel are you building for?\\
It is the version number string of your kernel. If you have given
the right source directory, it can normally find the string. This
should be exactly correct because if you have given a wrong version
string, the kernel may refuse to load the LiS modules.
\item Do you intend to run LiS on this machine with the currently running
kernel?\\
Normally you should choose yes.
\item How do you want to build STREAMS for the Linux kernel?\\
You can choose to make LiS a part of the kernel (linked with the kernel)
or a separate module that loads on your demand. However, you should
have no choice but make it a module. It is required by the Dialogic
System Release.
\item After building STREAMS, do you want kernel loadable modules installed?\\
I have chosen yes to let the make program move the modules into \texttt{/lib/modules/2.4.19/misc},
just because I don't want to run \texttt{make modules\_install} once
again in the kernel source.
\item Enter location of your kernel module directory\\
Give the location where the modules installed, usually it gives a
correct path as default.
\item When you make STREAMS, do you want to use backward compatible constants
in the file stropts.h?\\
I don't know much about this issue. I just used the default, UnixWare/Solaris
compatible instead of LiS backward compatible constants.
\item When you make STREAMS, do you want to use Solaris style cmn\_err?\\
As well, I don't know much about this issue. I just used the default,
SVR4 style.
\item When you make STREAMS, do you want to compile for source level debugging?\\
As I don't want tons of messages jamming my screen or the /var/log/message
file, I disabled it.
\item Do you want to use shared libraries?\\
I have tried to make it statically linked. But something mysterious
happened. I, threfore, recommends use the dynamic linking.
\end{enumerate}
That's all of the questions. After that, you should see the LiS compiles
and installs. In summary, if you are lucky enough (usually, in fact),
you just need to press enter for several times and use all the default
values would be okay. After LiS compiles, remember to run once again
the \texttt{make install} command to make the LiS helper scripts copied
into \texttt{/usr}.


\subsubsection{\label{RedHatSR}Obtain and Install System Release}

First of all, please go to Intel's web (http://resource.intel.com/telecom/support/releases/unix/index.html)
and download the {}``System Release 5.1 for Red Hat Linux 7.1/7.2
on Intel Architecture'' (SR 5.1). You will find a file named \texttt{LINUX\_SR5.1.tgz}
of 140 MB. Extract it into a directory, you will than find 15 RPM
files and 6 other files. If you are using RPM for package management,
just run \texttt{./install.sh} and you will see a menu of 6 items.
The first two are for SpringWare boards and the later four are for
DM3 boards. You just need to follow the instruction of menu and install
the required package.

For example, I am using a D/480JCT-1T1 card and therefore I choose
\texttt{1} to indicate that I am using a card of SpringWare series.
It will than show you a list of packages to be installed and install
the RPM packages by using the {}``\texttt{rpm -i}'' command. After
the installation is complete without error, you will see the menu
again and in this case, you can choose \texttt{Q} to quit. Further
details on installation can be obtained from Intel. There is a document
called {}``Installation and Configuration Guide for Linux'' at the
above URL describing how to do installations of the System Release
5.1 on RedHat (and alike systems).

Some remarks here: Unless you know specifically that you have \textit{Antares}
hardware, make sure not to select option 2 in the menu of install.sh.
That option will create some entries in your config file that will
clutter your screen with all kinds of mean, nasty, ugly messages during
initialization. (Thanks Gerry)

After the installation of the SR 5.1, configuration is required. The
configuration process is aimed to identify the model of board you
are using, the IRQ it should be used, create the kernel modules for
use as the device driver for the Dialogic devices and so on. Under
Red Hat-alike system, it is quite easy. You just go to the SR 5.1
directory where you just extracted the 140 MB tar ball, there is a
shell script called \texttt{config.sh}. Run it and you will see that
the first question asked is whether you have a DM3 type board. If
you answer no (that's me!), you will be prompted for whether you have
other type of Intel Dialogic Board. Answering these few question is
very important as wrong answer will give (may be) miserable results.

After these, you should follow the instructions appeared. It will
help you copy the required files into \texttt{/usr/src/LiS/} and build
a kernel modules (with installation to \texttt{/lib/modules/}\textit{version}\texttt{/misc/}).
It will also help creating configuration files in /usr/dialogic/conf/
by asking what models you have and what parameters (IRQs and board
IDs, for example) you want to use. Please follow sections 2.4 to 5.9
of the {}``Installation and Configuration Guide for Linux'' for
better descriptions.


\subsubsection{\label{TestSR}Testing if it works}

Rebooting is never a good practice for Linux systems! The real Linux
user only reboots at two occations: first is doing something with
the kernel, and the second is dealing with the hardware which means
you must stop the power. After installing the Dialogic SR, you don't
need to reboot. Just call the following commands:

\begin{lyxcode}
.~/etc/profile.d/orbacus.sh

.~/usr/dialogic/bin/setenv.sh

/etc/rc.d/init.d/orbacus~start

/etc/rc.d/init.d/dialogic~start
\end{lyxcode}
and you will have the same effect as rebooting the system.

All the Dialogic files relies in \texttt{/usr/dialogic/}, for example,
the executables are at \texttt{/usr/dialogic/bin/}. You can try go
there and call \texttt{dlstart} -- a script for starting the Dialogic
devices. If everything goes fine, it will download something onto
the Dialogic hardware and started some daemons with some diagnostic
information shown. After that, you can call the program \texttt{detect}
in the same directory which will than tell you about what's installed
in your system. You should normally see the model and board ID of
the telephony boards. If what I say is happened, congradulation! You've
make it. If not, try to read the Installation Guide or call the vendor
for help. Probably it is the problem of your kernel version or you
have chosen a wrong model during the post-install configuration. If
it still fails, try think your system as some scratch system and follow
the solution for Debian-alike distributions.


\subsection{Installation for Debian-alike (or non-RPM) distributions}

Seems that Intel belives Linux is equivalent to Red Hat. It looks
do not understand what Linux really is. However, frankly, I don't
like RedHat as many Linux gurus. Debian is my favoriate Linux favor,
but it do not have RPM given in parallel with Debian Package system.
I have no patience to install the RPM for it as well. Therefore, after
reading the {}``Installing Dialogic Boards in Slackware Linux''
by Diego Betancor on May 2001 (\texttt{http://diego.betancor.com/dialogicinlinux/}),
I decided to do it by hand.


\subsubsection{Prepare the kernel and Linux Streams}

Just as described in sections \ref{KernUp} and \ref{LiS}, you should
first keep the kernel up to date and with telephony options enabled.
Then give the Linux Streams installed as modules.


\subsubsection{Installing the Dialogic System Releases}

The first step is the same as the Red Hat-alike installation: please
get the 140 MB package first. Then extract it into some directory.
At there, just call the \texttt{install.sh} script. It should fail
definitely because we have no RPM system. But never mind, I just need
the output. The install script will tell you what packages you should
install. For example, in my case, I am installing a SpringWare board
and it told me that I should have \texttt{DLGCcom}, \texttt{DLGCcsp},
\texttt{DLGCdev}, \texttt{DLGCgc}, \texttt{DLGCooc}, \texttt{DLGCparms},
\texttt{DLGCpri} packages. Then, assuming you have the \texttt{rpm2cpio}
and \texttt{cpio} utilities, convert all the required RPMs into cpio
formats. In my case, I did this:

\begin{lyxcode}
for~pkgs~in~com~csp~dev~gc~ooc~parms~pri;~do

~~~~rpm2cpio~DLGC\$\{pkgs\}-{*}.i386.rpm~>~DLGC\$\{pkgs\}.cpio

done
\end{lyxcode}
Then, you have two choice. If you are safety-oriented, make a directory,
say, \texttt{/tmp/sr51/} and call the following command. If you are
risk-taking, just go directly to the root directory and take:

\begin{lyxcode}
for~pkgs~in~com~csp~dev~gc~ooc~parms~pri;~do

~~~~cpio~-id~<~/path/to/files/DLGC\$\{pkgs\}.cpio

done
\end{lyxcode}
Basically, if you call \texttt{cpio} in \texttt{/tmp/sr51}/, you will
see that there is a \texttt{./usr/dialogic/} directory and a \texttt{./etc/}
directory. Intel did not use the full power of RPM, but just make
it as an alternative of tar ball. So we just need to copy everything
into root (\texttt{cp -R {*} /}) is okay.

Debian is somewhat differs from Red Hat. Its start up script is at
\texttt{/etc/init.d/} instead of \texttt{/etc/rc.d/init.d/}, so you
have to move the dialogic startup script and other files to correct
locations. I would suggest to make a symbolic link as there is still
some scripts in \texttt{/usr/dialogic/bin} believes there is a \texttt{/etc/rc.d/init.d/orbacus}
script. Commands are as follows:

\begin{lyxcode}
ln~-s~/etc/rc.d/init.d/orbacus~/etc/init.d/orbacus

ln~-s~/etc/rc.d/rc3.d/S85orbacus~/etc/rc3.d/S85orbacus

ln~-s~/etc/rc.d/rc5.d/S85orbacus~/etc/rc5.d/S85orbacus
\end{lyxcode}
After that, you have to make the dynamic link libraries available
for the use of the SR 5.1 binaries and header files available for
the compilation of drivers. What you have to do is call the following
commands at anywhere:

\begin{lyxcode}
for~dyn~in~/usr/dialogic/lib/{*}~/usr/dialogic/ooc/lib/{*};~do

~~~~ln~-s~\$\{dyn\}~/usr/lib/`basename~\$\{dyn\}`

done

for~header~in~/usr/dialogic/inc/{*}.h;~do

~~~~ln~-s~\$\{header\}~/usr/include/`basename~\$\{header\}`

done
\end{lyxcode}
Now, you have created 58+105=163 soft links. Then run this list of
commands accordingly:

\begin{lyxcode}
ln~-s~/bin/pidof~/sbin/pidof

ln~-s~/usr/lib/libxerces\_c1\_5.so~/usr/lib/libxerces.so

mkdir~-p~/var/dialogic/log

chmod~766~/var/dialogic~/var/dialogic/log

ln~-s~/var/dialogic/log~/usr/dialogic/log

cp~/usr/dialogic/init.d/dialogic~/etc/init.d/dialogic

ln~-s~/etc/init.d/dialogic~/etc/rc3.d/S90dialogic

ln~-s~/etc/init.d/dialogic~/etc/rc5.d/S90dialogic

ln~-s~/etc/init.d/dialogic~/etc/rc0.d/K90dialogic

ln~-s~/etc/init.d/dialogic~/etc/rc6.d/K90dialogic

cp~/usr/dialogic/cfg/RtfConfig.xml~/usr/dialogic/cfg/DefaultRtfConfig.xml

chmod~-w~/usr/dialogic/cfg/DefaultRtfConfig.xml

echo~'.~/etc/profile.d/orbacus.sh'~>\,{}>~/etc/profile

echo~'.~/usr/dialogic/bin/setenv.sh'~>\,{}>~/etc/profile
\end{lyxcode}
which creates a symbolic links to make Debian compatible with the
Red Hat nomenclature and do some post-installation script that RPM
will do. Follows that, create a file \texttt{/usr/dialogic/cfg/.master.cfg}
with the following content (2 lines):

\begin{lyxcode}
00200~~vox~~sram~~1~~2

00201~~ant~~an~~~~1~~0
\end{lyxcode}
Now you have already finished most of the jobs. Next, is to go back
to source directory of the SR 5.1 which your \texttt{config.sh} locates,
and modify the \texttt{config.sh} script at line 20 where defines
variable \texttt{TEST\_PKG} to {}``\texttt{rpm -q}''. Because we
do not use rpm as the package management tool, the command \texttt{rpm
-q packagename} must fail and gives error return value. To make the
powerful \texttt{config.sh} script runs, we need the \texttt{\$TEST\_PKG
packagename} return successfully. You can, therefore, modify the \texttt{\$TEST\_PKG}
into anything that gives successful return values. What I did is make
it into {}``\texttt{echo}'', which will definitely return 0 whatever
you give.

After your modification, just execute \texttt{config.sh} and do the
configuration as in section \ref{RedHatSR} et voila!


\subsubsection{Problems?}

One thing very important: If your \texttt{/etc/init.d/orbacus} script
fails and says relocation error and do not understand the symbol \texttt{\_\_dynamic\_cast\_2},
it is not Debian's fault! The orbacus stuff are used to run on Red
Hat and it seems that the Red Hat's dynamic library is required. Two
possible solutions to this is (any one will work):

\begin{enumerate}
\item Use strip to kill the \texttt{\_\_dynamic\_cast\_2} symbol in \texttt{/usr/dialogic/ooc/lib/libOB.so},
command line is:\\
\texttt{strip -{}-strip-symbol=\_\_dynamic\_cast\_2 /usr/dialogic/oob/lib/libOB.so.4.0.5}
\item Backup your original library file and copy the file \texttt{/usr/lib/libstdc++-libc6.2-2.so.3}
from Red Hat (or Mandrake) distribution and overwrite the original
\end{enumerate}
The less-dirty solution is under investigation. Please give me some
time (and hints if you have).

Alright, back to the basic. After the installation, you should first
test the system according to section \ref{TestSR}. If it works, that's
great and skip to the next section for the works on Bayonne. If not,
probably you need to change something in \texttt{/usr/dialogic/cfg/}.
The most critical configuration file is \texttt{dialogic.cfg} (for
SpringWare only). In most case, you need to have a {}``\texttt{Dialog/HD=YES}''
in the section \texttt{{[}Genload - All Boards{]}}, which the \texttt{config.sh}
and \texttt{mkcfg} scripts always missed. For others parameters usable
in the \texttt{dialogic.cfg} file, you should consult section 3.3
of the Installation and Configuration Guide. The following is my configuration
file for the system with one VFX/PCI card for your reference:

\begin{lyxcode}
{[}Genload~-~All~Boards{]}

Dialog/HD=YES

BusType=SCBus

SCBusClockMaster=AUTOMATIC

SCBusClockMasterSource=AUTOMATIC



\#{[}Genload~-~ID~0{]}~/{*}~T1/E1~voice~HD~{*}/

\#Feature=dpd\_in

\#ParameterFile=spandti.prm



\#{[}Genload~-~ID~1{]}~/{*}~T1/E1~voice~HD~{*}/

\#ISDNProtocol=ne1
\end{lyxcode}
Sometimes, (in fact, most of the time) you may not have the file \texttt{/etc/rc.d/init.d/functions}
which the Red Hat-alike systems have. This function is useless but
give some fancy output on loading the start up scripts in \texttt{/etc/rc.d/init.d/{*}}.
If it is the case, your orbacus start up script will not run properly.
The solution is simple: Delete the lines (line 12, 47, 69) which imports
the \texttt{functions} script and call to \texttt{echo\_success} functions.


\section{Bayonne's matter}


\subsection{Making Bayonne from Scratch}

The main portal of Bayonne is at the GNU: \texttt{http://www.gnu.org/software/bayonne/}.
As described in the Bayonne User Manual, you needs four more components
to make Bayonne works, namely the CommonC++, ccAudio, ccRTP, ccScript.
These four components can be obtained via HTTP and FTP from ftp.gnu.org
or any other GNU FTP mirrors. At the time I write this document, the
latest versions available are: CommonC++2 1.0.1 (\texttt{commoncpp2-1.0.1.tar.gz}),
ccAudio 1.0 RC 1 (ccaudio-1.0rc1.tar.gz), ccRTP 0.9.1 (ccrtp-0.9.1.tar.gz),
ccScript 2.2.1 (\texttt{ccscript-2.2.1.tar.gz}). The latest version
of Bayonne is 1.0.0 (bayonne-1.0.0.tar.gz).

After obtaining all the tar balls, extract and install the CommonC++
first, then the ccAudio, ccRTP and ccScript in any other, and finally
extract and install the Bayonne package. The following screen dumps
shows what's being done:

\begin{lyxcode}
\#~ls~-l

bayonne-1.0.0.tar.gz~~~~~ccrtp-0.9.1.tar.gz~~~~~~~commoncpp2-1.0.1.tar.gz

ccaudio-1.0rc1.tar.gz~~~~ccscript-2.2.1.tar.gz

\#~for~x~in~{*};~do~tar~zxf~\$x;~done

\#~ls~-l

bayonne-1.0.0/~~~~~~~~~~~ccrtp-0.9.1/~~~~~~~~~~~~~commoncpp2-1.0.1/

bayonne-1.0.0.tar.gz~~~~~ccrtp-0.9.1.tar.gz~~~~~~~commoncpp2-1.0.1.tar.gz

ccaudio-1.0rc1/~~~~~~~~~~ccscript-2.2.1/

ccaudio-1.0rc1.tar.gz~~~~ccscript-2.2.1.tar.gz

\#~cd~commoncpp2-1.0.1

\#~./configure~-{}-prefix=/usr

....

\#~make~\&\&~make~install

....

\#~cd~../ccaudio-1.0rc1

\#~./configure~-{}-prefix=/usr

....

\#~make~\&\&~make~install

....

\#~cd~../ccrtp-0.9.1

\#~./configure~-{}-prefix=/usr

...

\#~make~\&\&~make~install

...

\#~cd~../ccscript-2.2.1

\#~./configure~-{}-prefix=/usr

...

\#~make~\&\&~make~install

...

\#~cd~../bayonne-1.0.0

\#~./configure~-{}-prefix=/usr

...

\#~make~\&\&~make~install

...
\end{lyxcode}
In some case, just as what I encountered, you may found some library
is missing and some packages do not compile. Basically it should not
happen but once it happens, give a symbolic link in \texttt{/usr/lib}
to fix the problem. Some \texttt{Makefile} just mix up the filename
of libraries for the linker. (I've failed to make bayonne 1.0rc1 because
the \texttt{libperl.so} do not exist but I have \texttt{libperl.so.5.6.1},
make a symbolic link will do.)

After the Bayonne and its depending packages are installed, you can
have a try to see if it works. Just start the Dialogic daemon (/usr/dialogic/init.d/dialogic),
and the Bayonne daemon (/etc/rc.d/init.d/bayonne). Then you will find
that a demo program starts. Dial to the telephony card and have a
try.


\subsection{Global Call}

Bayonne is not the only way to drive Dialogic devices. Intel provides
another suite of development software called Global Call for such
a purpose. Global Call is a suite of API, however, it is different
from the older, low-level C-API. The Global Call can make us share
the same code base between all different Dialogic devices, no matter
it is a DM3 or SpringWire board, it is digital or analog. Nevertheless,
it is not the purpose of discussing Global Call in this document.
If you are interested, please refer to http://resource.intel.com/telecom/support/releases/protocols/GCProtocols20/index.html
for details.


\section{Creating your own Bayonne Application}

Pending
\end{document}
